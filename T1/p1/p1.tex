\documentclass{article}

\usepackage{amssymb}
\usepackage{amsmath}
\usepackage{hyperref}

\begin{document}
    \section*{Explicación Solución}

    El problema corresponde a encontrar la cantidad de caminos de largo N en el siguiente grafo (se usa la matriz de adyacencia como representación):

    \begin{equation}\label{eq1}
        \begin{pmatrix}
            0 & 2 & 2 & 1\\
            2 & 0 & 0 & 1\\
            2 & 0 & 0 & 1\\
            1 & 1 & 1 & 0\\
        \end{pmatrix}
    \end{equation}
       

    Para facilitar el cálculo se busca una función recursiva de tres argumentos \(n\), \(u\) y \(v\), donde \(f(n,u,v)\) representa la cantidad de caminos de largo \(n\) que comienzan en u y terminan en \(v\) (se puede usar una función que solo considere donde terminan los caminos de largo \(n\)), con esto se ve que la respuesta para algún \(n\) en particular sería la suma de \(f(n,u,v)\) para cada \(u,v\in G\). Se nota que \(f(u,v,0)=1\) si y solo si \(u=v\), por lo que tenemos un caso base. Dado eso tambien notamos que dado un camino de largo \(n-1\) que comienza en \(u\) y termina en \(v'\), este se puede extender a un camino de largo \(n\) que comienza en \(u\) y termina en \(v\) si y solo si hay al menos una arista que conecte \(v'\) con \(v\), más aún la cantidad de formas de extender el camino es exactamente la cantidad de aristas entre \(v\)' y \(v\). Dado lo anterior podemos ver que \(f(n,u,v)\) es la suma de los \(\#e_{(u',v)}\cdot f(n-1,u,u')\), donde \(\#e_{(u',v)}\) es la cantidad de aristas entre \(u'\) y \(v\). Explicitamente se tiene la siguiente relación:

    \begin{equation}\label{eq2}
        f(n,u,v)=\sum_{u'\in G}\#e_{u',v}\cdot f(n-1,u,u').
    \end{equation}

    Con lo anterior vemos que hay una primera forma de calcular el resultado final, de forma top-down esto nos da una complejidad \(O(n)\) con una tabla \(O(n)\) elementos, para el siguiente pasó notemos que \(f(n,u,v)\) solo depende de los valores de \(f(n-1,u,u')\), en otras palabras si queremos calcular \(f(n,\cdot,\cdot)\) solo necesitamos tener precalculado \(f(n-1,\cdot,\cdot)\), por lo que usando un approach bottom-up llegamos a una complejidad \(O(n)\) con una tabla de \(O(1)\) elementos. Para el último paso es necesario darse cuenta de que la ecuación \eqref{eq2} es lineal, más aún podemos reexpresar \(f\) en terminos de una función \(g\) donde \(g(n)\) es una matriz donde \(f(n,u,v)=g(n)_{uv}\), usando esto llegamos a la expresión
    
    \begin{equation}\label{eq3}
        g(n+1)=A\cdot g(n),
    \end{equation}
    donde \(A\) es la matriz de \eqref{eq1}\footnote{Esto se puede verificar haciendo lo cálculos explicitos.}. Aquí se nota que, por inducción, se tiene que \(g(n)=A^n\cdot g(0)\), y recordando la definición de \(f(0,\cdot,\cdot)\), vemos que \(g(n)=A^n\). Ahora, se puede usar el algoritmo de exponenciación binaria para calcular \(g(n)\) con una complejidad \(O\log(n)\) y usando \(O(1)\) espacio extra.
\end{document}